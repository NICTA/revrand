\documentclass[11pt, oneside]{article}

% Packages
\usepackage{jmlr2e}
\usepackage{xcolor}
\usepackage[super]{nth}
\usepackage{mathtools, amsfonts}
\usepackage{graphicx}
\usepackage{subfig}

% Package config
% \sectionfont{\normalfont\sffamily\bfseries}
% \subsectionfont{\normalfont\sffamily\bfseries}
\hypersetup{colorlinks,
    linkcolor={red!50!black},
    citecolor={blue!50!black},
    urlcolor={blue!80!black}
}

% Notation and macros
%% Define bracket commands (normal, square and curly).
\newcommand{\brac} [1]  {\ensuremath{\left({#1}\right)}}
\newcommand{\sbrac}[1]  {\ensuremath{\left[{#1}\right]}}
\newcommand{\cbrac}[1]  {\ensuremath{\left\{{#1}\right\}}}
\newcommand{\abrac}[1]  {\ensuremath{\left\langle{#1}\right\rangle}}


%% Symbols

% General
\newcommand{\test}       {\ensuremath{^{*}}}
\newcommand{\testT}      {\ensuremath{^{*\top}\!}}
\newcommand{\ttest}      {\ensuremath{^{**}}}
\newcommand{\real}   [1] {\ensuremath{\mathbb{R}^{#1}}}
\newcommand{\natu}   [1] {\ensuremath{\mathbb{N}^{#1}}}
\newcommand{\ident}  [1] {\ensuremath{\mathbf{I}_{#1}}}

% Variables
\newcommand{\targs}     {\ensuremath{y}}
\newcommand{\targ}      {\ensuremath{\mathbf{y}}}
\newcommand{\targd}     {\ensuremath{\mathbf{y}\test}}
\newcommand{\targdT}    {\ensuremath{\mathbf{y}\testT}}
\newcommand{\ins}       {\ensuremath{\mathbf{X}}}
\newcommand{\inss}      {\ensuremath{\mathbf{x}}}
\newcommand{\locs}      {\ensuremath{\mathbf{z}}}
\newcommand{\minibatch} {\ensuremath{\mathbf{B}}}
\newcommand{\feat}      {\ensuremath{\boldsymbol\Phi}}
\newcommand{\feats}     {\ensuremath{\boldsymbol\phi}}
\newcommand{\featfunc}  {\ensuremath{\phi}}
\newcommand{\weights}   {\ensuremath{\mathbf{w}}}
\newcommand{\var}       {\ensuremath{\sigma^2}}
\newcommand{\pomean}    {\ensuremath{\mathbf{m}}}
\newcommand{\pocov}     {\ensuremath{\mathbf{C}}}
\newcommand{\dpocov}    {\ensuremath{\boldsymbol{\Psi}}}
\newcommand{\dpocovs}   {\ensuremath{\Psi}}
\newcommand{\jacob}[1]  {\ensuremath{\mathbf{J}_{#1}}}
\newcommand{\hess}[1]   {\ensuremath{\mathbf{H}_{#1}}}
\newcommand{\reg}       {\ensuremath{\lambda}}
\newcommand{\param}     {\ensuremath{\theta}}
\newcommand{\lparam}    {\ensuremath{\gamma}}
\newcommand{\hyper}     {\ensuremath{\alpha}}
\newcommand{\lrate}     {\ensuremath{\eta}}

% Gaussian Process
\newcommand{\kernl}     {\ensuremath{k}}
\newcommand{\Kernl}     {\ensuremath{\mathbf{k}}}
\newcommand{\KERNL}     {\ensuremath{\mathbf{K}}}


%% Operations
\newcommand{\T}          {\ensuremath{^{\!\top}}}
\newcommand{\inv}        {\ensuremath{^{\text{-}1}}}
\newcommand{\deter}[1]   {\ensuremath{\left|{#1}\right|}}
\newcommand{\trace}[1]   {\ensuremath{\text{tr}\!\brac{#1}}}
\newcommand{\diag}[1]    {\ensuremath{\text{diag}\!\brac{#1}}}
\newcommand{\expec}[2]   {\ensuremath{\abrac{#2}_{\!{#1}}}}
\newcommand{\expece}[2]  {\ensuremath{\mathbb{E}_{#1}\!\sbrac{#2}}}
\newcommand{\evar} [2]   {\ensuremath{\mathbb{V}_{#1}\!\sbrac{#2}}}
\newcommand{\KL}[2]      {\ensuremath{\text{KL}\!\sbrac{{#1}\!\parallel\!{#2}}}}
\newcommand{\entropy}[1] {\ensuremath{\mathbb{H}\sbrac{#1}}}
\newcommand{\lnorm}[2]   {\ensuremath{\left\|{#2}\right\|_{{#1}}}}


%% Functions, PDFs etc
\newcommand{\func} [3] {\ensuremath{{#1}_{#3}\!\brac{{#2}}}}
\newcommand{\ffunc} [2] {\func{f}{#1}{#2}}
\newcommand{\activ} [1] {\func{g}{#1}{}}
\newcommand{\Prob}  [1] {\ensuremath{P\!\brac{#1}}}
\newcommand{\ProbC} [2] {\ensuremath{P\!\left({#1}\middle\vert{#2}\right)}}
\newcommand{\iProb}  [1] {\ensuremath{P\inv\!\brac{#1}}}
\newcommand{\iProbC} [2] {\ensuremath{P\inv\!\left({#1}\middle\vert{#2}\right)}}
\newcommand{\quant}  [1] {\ensuremath{Q\!\brac{#1}}}
\newcommand{\prob}  [1] {\ensuremath{p\!\brac{#1}}}
\newcommand{\probC} [2] {\ensuremath{p\!\left({#1}\middle\vert{#2}\right)}}
\newcommand{\qrob}  [1] {\ensuremath{q\!\brac{#1}}}
\newcommand{\qrobC} [2] {\ensuremath{q\!\left({#1}\middle\vert{#2}\right)}}
\newcommand{\gaus}  [1] {\ensuremath{\mathcal{N}\!\brac{#1}}}
\newcommand{\gausC} [2] {\ensuremath{\mathcal{N}\!\left({#1}\middle\vert{#2}\right)}}
\newcommand{\bern}  [1] {\ensuremath{\textrm{Bern}\!\brac{#1}}}
\newcommand{\bernC} [2] {\ensuremath{\textrm{Bern}\!\left({#1}\middle\vert{#2}\right)}}
\newcommand{\kfunc} [2] {\ensuremath{\kernl\!\brac{{#1}, {#2}}}}
\newcommand{\expon} [2] {\ensuremath{{#1}\!\times\!10^{#2}}}
\newcommand{\bigo}  [1] {\ensuremath{\mathcal{O}\!\brac{{#1}}}}
\newcommand{\elbo}      {\ensuremath{\mathcal{L}}}


%% Operators
\DeclareMathOperator*{\argmax}{\operatorname*{argmax}}
\DeclareMathOperator*{\argmin}{\operatorname*{argmin}}


% Preamble

% Heading arguments are
% {volume}{year}{pages}{submitted}{published}{author-full-names}
% \jmlrheading{}{}{}{}{}{}

% Short headings should be running head and authors last names
\ShortHeadings{\revrand{}~Technical Report}{Steinberg, Tiao, McCalman, Reid and
    O'Callaghan}
\firstpageno{1}

\title{\revrand{}: Technical Report}
\author{\name Daniel Steinberg \email daniel.steinberg@data61.csiro.au \\
        \name Louis Tiao \email louis.tiao@data61.csiro.au \\
        \name Lachlan McCalman \email lachlan.mccalman@data61.csiro.au \\
        \name Alistair Reid \email alistair.reid@data61.csiro.au \\
        \name Simon O'Callaghan \email simon.ocallaghan@data61.csiro.au \\
        \addr DATA61, CSIRO \\
        Sydney, Australia}

\date{}

\begin{document}

\maketitle
% \vspace{-0.5cm}
% \noindent\makebox[\linewidth]{\rule{\linewidth}{0.8pt}}
% \vspace{0.3cm}

\begin{abstract}
    This is a technical report on the \revrand{} software library. This
    library implements Bayesian linear models (Bayesian linear regression),
    generalized linear models and approximate Gaussian processes. These
    algorithms have been implemented such that they can be used for large-scale
    learning by using stochastic variational inference. All of the algorithms
    in \revrand{} use a unified feature composition framework that allows
    for easy concatenation and selective application of regression basis
    functions.
\end{abstract}

\tableofcontents

\section{Core Algorithms}

Recent developments in stochastic gradient optimisation have simplified the
application of machine learning to massive datasets, in particular, modern
stochastic optimisation algorithms are far more robust to initial learning rate
settings (see \citet{kingma2014adam} for instance). Furthermore, Bayesian
machine learning algorithms have a number of well defined methods for learning
model parameters and \emph{hyperparameters} from training data, that do not
involve cross validation. When used in combination, stochastically optimized
Bayesian machine learning algorithms allow practitioners to learn probabilistic
predictors from large data sets with minimal tuning and retraining.

We make use of some of these recent developments in stochastic gradient methods
and stochastic variational inference in \revrand{} for supervised regression
tasks. We outline the core algorithms implemented in \revrand{} in this section
of the report, beginning with how we use stochastic optimization with
variational objectives, then onto the two core regression algorithms in
\revrand{}, and how to use them to appropriate Gaussian Processes (GPs).

\subsection{Stochastic Gradients and Variational Objective Functions}
\label{sub:stochvar}

When a machine learning objective function factorises over data,
\begin{equation}
    \ffunc{\ins, \param}{} = \sum_{\inss \in \ins} \ffunc{\inss, \param}{},
\end{equation}
a regular gradient descent algorithm would perform the following iterations to
minimise the function w.r.t.\ the parameters $\param$,
\begin{equation}
    \param_k := \param_{k-1} - \lrate_k \sum_{\inss \in \ins}
    \nabla_{\param} \ffunc{\inss, \param}{}\!|_{\param = \param_{k-1}},
\end{equation}
where $\lrate_k$ is the learning rate (step size) at iteration $k$. Stochastic
gradient algorithms use the following update,
\begin{equation}
    \param_k := \param_{k-1} - \lrate_k \sum_{\inss \in \minibatch}
    \nabla_{\param} \ffunc{\inss, \param}{}\!|_{\param = \param_{k-1}},
    \label{eq:sg}
\end{equation}
where $\minibatch \subset \ins$ is a mini-batch of the original dataset, where
$|\minibatch| \ll |\ins|$. Frequently objective functions to not entirely
decompose over the data, i.e.,
\begin{equation}
    \ffunc{\ins, \param}{} = \sum_{\inss \in \ins} \ffunc{\inss, \param}{}
    + \func{g}{\param}{}.
    \label{eq:objwconst}
\end{equation}
However, it is trivial to make these objectives work in a stochastic gradient
setting. Let $M = |\minibatch|$ and $N = |\ins|$, then we can divide the
contribution of the constant term amongst the mini-batches in stochastic 
gradients,
\begin{equation}
    \param_k := \param_{k-1} - \lrate_k \sum_{\inss \in \minibatch}
    \nabla_{\param} \ffunc{\inss, \param}{}\!|_{\param = \param_{k-1}}
    - \frac{M}{N} \lrate_k \nabla_{\param} 
    \func{g}{\param}{}\!|_{\param = \param_{k-1}}.
    \label{eq:wsg}
\end{equation} 
or, equivalently, boost the contribution of the mini-batch,
\begin{equation}
    \param_k := \param_{k-1} - \frac{N}{M} \lrate_k \sum_{\inss \in \minibatch}
    \nabla_{\param} \ffunc{\inss, \param}{}\!|_{\param = \param_{k-1}}
    - \lrate_k \nabla_{\param} 
    \func{g}{\param}{}\!|_{\param = \param_{k-1}}.
    \label{eq:wsg2}
\end{equation} 
This is particularly relevant for variational inference where the evidence
lower bound objective has a component independent of the data. For example, 
lets consider the model,
\begin{align}
    \text{Likelihood:} \quad &\prod^N_{n=1} \probC{\targs_n}{\param}, \\
    \text{prior:} \quad &\probC{\param}{\hyper},
\end{align}
where we want to learn the values of the hyper-parameters, $\hyper$. Minimising
negative log-marginal likelihood is a reasonable objective in this
instance\footnote{see \citet[Chapter 3.4]{bishop2006pattern} and \citet[Chapter
    5]{Rasmussen2006} for a discussion on this.}, since we don't care about the
value(s) of $\param$,
\begin{equation}
    \argmin_\hyper - \log \int \prod^N_{n=1} \probC{\targs_n}{\param}
    \probC{\param}{\hyper} d \param.
    \label{eq:lml}
\end{equation}
There are two problems with this objective however, (1) it may not in general factor over
data and (2) the integral may be intractable, for instance, if the prior and
likelihood are not conjugate. In variational inference we use Jensen's
inequality to lower-bound log-marginal likelihood with a tractable objective
function called the evidence lower bound (ELBO),
\begin{align}
    \log \probC{\targ}{\hyper} =& \log \int 
        \prod^N_{n=1} \probC{\targs_n}{\param} 
        \probC{\param}{\hyper} d \param \nonumber \\
        =& \log \int 
        \frac{\prod_n \probC{\targs_n}{\param} \probC{\param}{\hyper}}
        {\qrob{\param}} \qrob{\param} d \param \nonumber \\
        \geq& \int \qrob{\param} \log \sbrac{%
            \frac{\prod_n \probC{\targs_n}{\param} 
            \probC{\param}{\hyper}}{\qrob{\param}}}
        d \param
\end{align}
where $\qrob{\param}$ is an approximation of the true posterior
$\probC{\param}{\cbrac{\targs_1, \ldots, \targs_N}, \hyper}$, chosen to make 
inference easier. This can be re-written as,
\begin{equation}
    \elbo = \sum^N_{n=1} \expec{q}{\log\probC{\targs_n}{\param}} -
    \KL{\qrob{\param}}{\probC{\param}{\hyper}},
    \label{eq:elbo}
\end{equation}
which takes the form of Equation~\eqref{eq:objwconst}, and so if we use
stochastic gradients optimisation we can weight the Kullback-Leibler term like
the constant term, $\func{g}{\cdot}{}$, from Equation~\eqref{eq:wsg}, or boost
the expected log likelihood term like in Equation~\eqref{eq:wsg2}.


\subsection{Bayesian Linear Regression -- \texttt{StandardLinearModel}}

The first machine learning algorithm in \revrand{} is a simple Bayesian linear
regressor of the following form,
\begin{align}
    \text{Likelihood:} \quad &\prod^N_{n=1} 
    \gausC{\targs_n}{\feats_n\T\weights, \var}, \label{eq:slm_like}\\
    \text{prior:} \quad &\gausC{\weights}{\mathbf{0}, \Reg},
    \label{eq:slm_prior}
\end{align}
where $\feats_n := \func{\featsym}{\inss_n, \param}{}$ is a feature, or basis,
function that maps $\featsym : \real{d} \to \real{D}$, and $\Reg \in
\real{D\times D}$ is a diagonal matrix (i.e.~it could be $\reg\ident{D}$) that
has the effect of regularising the magnitude of the weights. This is the same
algorithm described in~\citet[Chapter 2]{Rasmussen2006}. We then:
\begin{itemize}
    \item Optimise $\var, \Reg$ and $\param$ w.r.t.\ log-marginal likelihood,
        \begin{equation}
            \log \probC{\targ}{\var, \Reg, \param} =
            \log \gausC{\targ}{\mathbf{0}, \var\ident{N} + \feat\T\Reg\feat},
        \end{equation}
        where $\feat \in \real{N \times D}$ is the concatenation of all the
        features, $\feats_n$. Note this results in the covariance of the
        log-marginal likelihood being $N \times N$, though we can use the
        Woodbury identity to simplify the corresponding matrix inversion.
    \item Solve analytically for the posterior over weights, $\weights | \targ
        \sim \gaus{\pomean, \pocov}$ given the above hyperparameters, where,
        \begin{align*}
            \pocov &= \sbrac{\Reg\inv + \frac{1}{\var}
                \feat\T \feat}\inv, \\
            \pomean &= \frac{1}{\var} \pocov \feat\T \targ.
        \end{align*}
    \item Use the predictive distribution
        \begin{align}
            \probC{\targs\test}{\targ, \ins, \inss\test} &= \int
            \gausC{\targs\test}{\feats\testT\weights, \var}
            \gausC{\weights}{\pomean, \pocov} d\weights, \nonumber \\
            &= \gausC{\targs\test}{\feats\testT \pomean,
                \var + \feats\testT \pocov \feats\test}
        \end{align}
        for query inputs, $\inss\test$. This gives us the useful expectations,
        \begin{align}
            \expece{}{\targs\test} &= \feats\testT\pomean, \\
            \evar{}{\targs\test} &= \var + \feats\testT\pocov\feats\testT.
        \end{align}
\end{itemize}

It is actually easier to use the ELBO form with stochastic gradients for
learning the parameters of this algorithm, rather than log-marginal likelihood
recast using the Woodbury identity.  This is because it is plainly in the same
form as Equation \eqref{eq:objwconst}, though it would give the same result as
log-marginal likelihood because the ``approximate'' posterior is the same form
as the true posterior, i.e.\ $\qrob{\weights} = \gausC{\weights}{\pomean,
    \pocov}$. The ELBO for this model is,
\begin{equation}
    \elbo = \sum^N_{n=1} 
    \expec{q}{\log\gausC{\targs_n}{\feats_n\T\weights, \var}}
    - \KL{\gausC{\weights}{\pomean, \pocov}}
        {\gausC{\weights}{\mathbf{0}, \Reg}}.
    \label{eq:slmobj}
\end{equation}
More specifically,
\begin{align*}
    \expec{q}{\log\gausC{\targs_n}{\feats_n\T\weights, \var}} &=
    \log \gausC{\targs_n}{\feats_n\T\pomean, \var}
    - \frac{1}{2 \var} \trace{\feats_n\T\feats_n\pocov}, \\
    \KL{\gausC{\weights}{\pomean, \pocov}}
        {\gausC{\weights}{\mathbf{0}, \Reg}} &=
        \frac{1}{2} \sbrac{\trace{\Reg\inv\pocov} + \pomean\T\Reg\inv\pomean 
        -  \log\deter{\pocov} + \log\deter{\Reg} - D}
\end{align*}
We have not implemented a stochastic gradient version of this algorithm since
it still requires a matrix solve of a $D \times D$ matrix, and so is \bigo{D^3}
in complexity, per iteration. This is true even if we optimise the posterior
covariance directly (or a triangular parameterisation). The GLM presented in
the next section circumvents this issue, and is more suited to really large $N$
and $D$ problems.


\subsection{Bayesian Generalized Linear Models --
    \texttt{GeneralizedLinearModel}}

The algorithm of primary interest in \revrand{} is the Bayesian generalized
linear model (GLM). The general form of the model is,
\begin{align}
    \text{Likelihood:} \quad &\prod^N_{n=1} 
        \probC{\targs_n}{\activ{\feats_n\T\weights}, \lparam}, 
        \label{eq:glm_like} \\
    \text{prior:} \quad &\gausC{\weights}{\mathbf{0}, \Reg},
        \label{eq:glm_prior}
\end{align}
for an arbitrary univariate likelihood, $\prob{\cdot}$, with an appropriate
transformation (inverse link) function, $\activ{\cdot}$, and parameter(s),
$\lparam$. 

Naturally, both calculating the exact posterior over the weights,
$\probC{\weights}{\targ, \ins}$, and the log-marginal likelihood,
$\prob{\targ}$, for hyperparameter learning are intractable since we may have a
non-conjugate relationship between the likelihood and prior. Therefore we must
resort to approximating the true posterior and the log-marginal likelihood.

Firstly, we approximate the true posterior over weights with a mixture of $K$
diagonal Gaussians,
\begin{align}
    \probC{\weights}{\targ, \ins} &\approx \qrob{\weights}, \nonumber \\
    &= \frac{1}{K} \sum^K_{k=1} \gausC{\weights}{\pomean_k, \dpocov_k},
\end{align}
where $\dpocov_k = \diag{[\dpocovs_{k,1}, \ldots, \dpocovs_{k, D}]\T}$, which
is inspired by similar approximations made in \citet{gershman2012} and 
\citet{nguyen2014automated}. This is a very flexible form for the approximate
posterior, and has the nice consequence that our algorithm will longer have a
\bigo{D^3} cost associated with that of a full Gaussian covariance.

Then we approximate the log marginal likelihood using variational Bayes with
the re-parameterisation trick \citep{kingma2014auto}. The exact lower bound on
log marginal likelihood is,
\begin{equation}
    \elbo = \sum^N_{n=1} 
    \expec{q}{\log\probC{\targs_n}{\activ{\feats_n\T\weights}, \lparam}}
    - \KL{\textstyle\frac{1}{K}\textstyle\sum_{k}
            \gausC{\weights}{\pomean_k, \dpocov_k}}
        {\gausC{\weights}{\mathbf{0}, \Reg}}.
    \label{eq:glmobj}
\end{equation}
This can be expanded,
\begin{multline}
    \elbo = \frac{1}{K} \sum^K_{k=1} \sum^N_{n=1} 
    \expec{q_k}{\log\probC{\targs_n}
        {\activ{\feats_n\T\weights}, \lparam}}
    + \frac{1}{K} \sum^K_{k=1}
        \expec{q_k}{\log\gausC{\weights}{\mathbf{0}, \Reg}} \\
    + \entropy{\textstyle\frac{1}{K}\textstyle\sum_{k}
            \gausC{\weights}{\pomean_k, \dpocov_k}},
    \label{eq:glmobj_exp}
\end{multline}
but unfortunately there are two intractable integrals here: the expected log
likelihood, and the entropy of the Gaussian mixture. We can use the lower bound
on the entropy term also used in \citet{gershman2012, nguyen2014automated},
\begin{equation}
    \entropy{\textstyle\frac{1}{K}\textstyle\sum_{k}
        \gausC{\weights}{\pomean_k, \dpocov_k}} \geq
    - \frac{1}{K} \sum_{k=1}^K \log \sum_{j=1}^K \frac{1}{K}
    \gausC{\pomean_k}{\pomean_j, \dpocov_k + \dpocov_j},
\end{equation}
and then we use the re-parameterisation trick in \citet{kingma2014auto} to 
obtain samples of the expected log likelihood,
\begin{equation}
    \sum^N_{n=1} 
    \expec{q_k}{\log\probC{\targs_n}
        {\activ{\feats_n\T\weights}, \lparam}} \approx
    \frac{1}{L} \sum^{L}_{l=1} \sum^N_{n=1} \log\probC{\targs_n}
    {\activ{\feats_n\T\hat{\weights}_k^{\brac{l}}}, \lparam}
\end{equation}
where,
\begin{equation}
    \hat{\weights}_k^{\brac{l}} =
    \func{f}{\pomean_k, \dpocov_k, \resamp^{(l)}}{k} =
    \pomean_k + \sqrt{\dpocov_k} \odot \resamp^{(l)},
    \qquad \resamp^{(l)} \sim \gaus{\mathbf{0}, \ident{D}}.
\end{equation}
Here $\odot$ is the element-wise product. We can also use this trick to obtain
low variance samples of the derivatives for the GLM's parameters and
hyperparameters, $\frac{\partial}{\partial \alpha} \expec{q(\alpha)}{\log
    \probC{\targs}{\alpha}} \approx \frac{1}{L} \sum^L_{l=1}
\frac{\partial}{\partial\alpha} \log \probC{\targs}{\func{f}{\alpha,
        \resamp^{(l)}}{}}$, which simplifies the implementation greatly! The
final auto-encoding variational Bayes objective for our GLM is,
\begin{multline}
    \elbo \approx \frac{1}{K} \sum^K_{k=1} \Bigg[
    \frac{1}{L} \sum^L_{l=1} \sum^N_{n=1} 
    \log\probC{\targs_n}
    {\activ{\feats_n\T \hat{\weights}_k^{\brac{l}}}, \lparam}
    + \log\gausC{\pomean_k}{\mathbf{0}, \Reg}
    - \frac{1}{2} \trace{\Reg\inv\dpocov_k} \\
    - \log \sum_{j=1}^K \frac{1}{K}
    \gausC{\pomean_k}{\pomean_j, \dpocov_k + \dpocov_j}
    \Bigg].
    \label{eq:glmobj_exp}
\end{multline}
We can straight-forwardly use this objective with a stochastic optimization
and Equation \eqref{eq:wsg} or \eqref{eq:wsg2}.

The most simple and accurate method for approximating the predictive
distribution, $\probC{\targs\test}{\targ, \ins, \inss\test}$ is to Monte-Carlo
sample the integral,
\begin{equation}
    \probC{\targs\test}{\targ, \ins, \inss\test} \approx
    \int \probC{\targ}{\activ{\feats\testT\weights}, \lparam}
    \frac{1}{K} \sum^K_{k=1} \gausC{\weights}{\pomean_k, \dpocov_k} d \weights.
    \label{eq:glppred}
\end{equation}
However, this integral is not particularly useful unless we wish to evaluate
known $\targ\test$ under the model. For prediction, it is more useful to
compute (using Monte-Carlo integration) the predictive expectation, 
\begin{align}
    \expece{}{\targs\test} \approx&
    \int \frac{1}{K} \sum^K_{k=1} \gausC{\weights}{\pomean_k, \dpocov_k}
    \int \targs\test \probC{\targs\test}{\activ{\feats\testT\weights}, \lparam}
    d \targs\test d \weights
    \nonumber\\
    =& \int \expece{}{\probC{\targs\test}
        {\activ{\feats\testT\weights}, \lparam}}
    \frac{1}{K} \sum^K_{k=1} \gausC{\weights}{\pomean_k, \dpocov_k}
    d \weights.
    \label{eq:glmexpec}
\end{align}

Often we find $\expece{}{\probC{\targs\test}{\activ{\feats\testT\weights},
        \lparam}} = \activ{\feats\testT\weights}$, however this is only true
with with right choice of activation function\footnote{That is, if we choose
    the activation function corresponding to the canonical link function.}.
Furthermore, it is useful to compute quantiles of the predictive density in
order to ascertain the predictive uncertainty. We start by sampling the
predictive cumulative density function, $\Prob{\cdot}$,
\begin{align}
    &\ProbC{\targs\test \leq \cdfAlph}{\targ, \ins, \inss\test} \nonumber\\ 
    &\qquad\approx \int 
    \frac{1}{K} \sum^K_{k=1} \gausC{\weights}{\pomean_k, \dpocov_k}
    \int^{\cdfAlph}_{-\infty} 
    \probC{\targs\test}{\activ{\feats\testT\weights}, \lparam}
    d \targs\test d \weights \nonumber\\
    &\qquad= \int
    \ProbC{\targs\test \leq \cdfAlph}{\activ{\feats\testT\weights}, \lparam}
    \frac{1}{K} \sum^K_{k=1} \gausC{\weights}{\pomean_k, \dpocov_k}
    d \weights.
    \label{eq:lapexpec}
\end{align}
Once we have obtained sufficient samples from the (mixture) posterior we can
obtain quantiles, $\cdfAlph$, for some chosen level of probability, $p$, using
root finding techniques. Specifically, we use root finding techniques to solve
the following for $\cdfAlph$,
\begin{equation}
    \ProbC{\targs\test \leq \cdfAlph}{\targ, \ins, \inss\test} - p = 0.
\end{equation}

\subsection{Large Scale Gaussian Process Approximation}

In \revrand{} we approximate Gaussian Processes \citep{Rasmussen2006} with
our standard and generalized linear models by using random feature functions
such as those of \citeauthor{rahimi2007} \citeyearpar{rahimi2007,rahimi2008}
and \cite{le2013fastfood}. They use Bochner's theorem regarding the
relationship between a kernel and the Fourier transform of a non-negative
measure (e.g.~a probability measure) that establishes the duality of the
covariance function of a stationary process and its spectral density,
\begin{align}
	\kernl(\tfourier) &= \int \specfourier(\ffourier) 
    e^{i \ffourier\T  \tfourier} d \ffourier,  \\
	\specfourier(\ffourier) &= \int \kernl(\tfourier) 
    e^{- i \ffourier\T \tfourier}  d \tfourier.
	\label{eq:fourier}
\end{align}
where $\kernl(\cdot)$ is a kernel function, and $\specfourier(\cdot)$ its
spectral density. \citeauthor{rahimi2007}'s  main insight
\citeyearpar{rahimi2007} is that we can approximate the kernel by constructing
`suitable' random features and Monte Carlo averaging over samples from
$\specfourier(\ffourier)$ for \emph{shift invariant} kernels,
\begin{equation}
    \kernl(\inss - \inssprime) = \kernl(\tfourier) 
    \approx \frac{1}{D} \sum_{i=1}^{D} \singlefeatfunc{\inss}{i}\T\!
	\singlefeatfunc{\inssprime}{i},
	\label{eq:mcapprox}
\end{equation}
where $\singlefeatfunc{\inss}{i}$ corresponds to the $i$th sample from the
feature map. An example of a radial basis kernel feature vector construction
using the above approximation is,
\begin{align}
	\nonumber
    \sbrac{\singlefeatfunc{\inss}{i}, \singlefeatfunc{\inss}{D+i}}&= 
    \frac{1}{\sqrt{D}} \sbrac{\cos(\ffourier_i^T \inss), 
    \sin(\ffourier_i^T \inss)}, \qquad \\
    \text{with}~\ffourier_i & \sim 
    \gausC{\ffourier_i}{\mathbf{0}, \lenscale\inv \ident{d}},
\end{align}
for $i=1, \ldots, D$,  which in fact is a mapping into a $2 D$-dimensional
feature space. See Table~\ref{tab:randommappings} for some of the random kernel
approximations we use in \revrand{}.

\begin{table}[htb]

    \centering

    \caption{Kernels and the corresponding Fourier weight ($\ffourier_i$)
        sampling distributions for the \citet{rahimi2007}-style random bases in
        \revrand{}. Here GAL refers to a multivariate Laplace
        distribution~\cite{kozubowski2013multivariate}.}

    \begin{tabular}{r|c c}
        \textbf{Kernel} & $\kernl\!\brac{\inss - \inssprime}$
        & $\specfourier(\ffourier)$ \\
        \hline
        RBF &
        $\exp\brac{-\frac{\lnorm{}{\inss - \inssprime}^2}{2\lenscale^2}}$ &
            $\ffourier_i \sim \gaus{\mathbf{0}, \lenscale\inv\ident{d}}$, \\
        Laplace & $\exp\brac{-\frac{\lnorm{}{\inss-\inssprime}}{\lenscale}}$ &
            $\ffourier_i \sim \prod_d \text{Cauchy}\!\brac{\lenscale\inv}$ \\
        Cauchy &
            $\frac{1}{1+\brac{\lnorm{}{\inss-\inssprime}/\lenscale}^2}$ & 
            $\ffourier_i \sim \text{GAL}\brac{1, \mathbf{0},
                \lenscale\inv\ident{d}}$ \\
        Matern $3/2$ & 
            $\brac{1 + \frac{\sqrt{3}\lnorm{}{\inss-\inssprime}}{\lenscale}}
            \exp\brac{-\frac{\sqrt{3}\lnorm{}{\inss-\inssprime}}{\lenscale}}$ & 
            $\ffourier_i \sim 
            \text{t}_{\nu=3}\!\brac{\mathbf{0}, \lenscale\inv\ident{d}}$ \\
        Matern $5/2$ &
            $\brac{1 + \frac{\sqrt{5}\lnorm{}{\inss-\inssprime}}{\lenscale} + 
                \frac{5\lnorm{}{\inss - \inssprime}^2}{3\lenscale^2}}
            \exp\brac{-\frac{\sqrt{5}\lnorm{}{\inss-\inssprime}}{\lenscale}}$ & 
            $\ffourier_i \sim 
            \text{t}_{\nu=5}\!\brac{\mathbf{0}, \lenscale\inv\ident{d}}$ \\
        \hline
    \end{tabular}

    \label{tab:randommappings}

\end{table}

All of the kernel functions in Table \ref{tab:randommappings} have length scale
hyperparameters, and in \revrand{} these can be isotropic or anisotropic length
scales that are learned alongside the other hyperparameters. We have also
implemented a few other variants of these random basis functions, namely the
Fastfood \citep{le2013fastfood}, A la Carte \citep{yang2014} and orthogonal
random features \citep{Yu2016orthogonal} that improve either scalability,
representational flexibility respectively or both. In Figure \ref{fig:kerns} we
compare the basis approximations with their corresponding kernels. We refer the
reader to our demo notebooks for further discussion and experiments comparing
these approximate basis functions and the kernels they approximate.

\begin{figure}[tbp]
    \centering
    \subfloat[][RBF]{
        \includegraphics[width=0.4\linewidth]{figs/rbf.png}
        \label{fig:rbf}
    }
    \subfloat[][RBF Fastfood]{
        \includegraphics[width=0.4\linewidth]{figs/rbfFF.png}
        \label{fig:rbfFF}
    }\\
    \subfloat[][Laplace]{
        \includegraphics[width=0.4\linewidth]{figs/laplace.png}
        \label{fig:laplace}
    }
    \subfloat[][Cauchy]{
        \includegraphics[width=0.4\linewidth]{figs/cauchy.png}
        \label{fig:cauchy}
    }\\
    \subfloat[][Matern 3/2]{
        \includegraphics[width=0.4\linewidth]{figs/matern32.png}
        \label{fig:matern32}
    }
    \subfloat[][Matern 5/2]{
        \includegraphics[width=0.4\linewidth]{figs/matern52.png}
        \label{fig:matern52}
    }\\
    \subfloat[][RBF Orthogonal]{
        \includegraphics[width=0.4\linewidth]{figs/rbforth.png}
        \label{fig:orth}
    }
    \subfloat[][RBF + Laplace]{
        \includegraphics[width=0.4\linewidth]{figs/combo.png}
        \label{fig:combo}
    }\\
    \caption{Comparison of various kernels and their basis approximations. In
        each of these figures 1500 random bases have been used, and the inputs
        to the kernels are 10 dimensional, $\inss \in \real{10}$.}
    \label{fig:kerns}
\end{figure}

An important feature of \revrand{} is that we can combine these random features
(and non-random features) to build even more expressive features. In
particular, we support basis \emph{concatenation} while still allowing basis
function hyperparameter learning,
\begin{equation}
    \feat_{\text{cat}} = \sbrac{
        \func{\featsym}{\ins, \param_1}{1},
        \func{\featsym}{\ins, \param_2}{2},
        \ldots,
        \func{\featsym}{\ins, \param_P}{P}
    }.
\end{equation}
In the following manner this approximates kernel addition,
\begin{equation}
    \reg_1 \kernl_1\!\brac{\ins, \ins, \param_1} +
    \reg_2 \kernl_2\!\brac{\ins, \ins, \param_2} + \ldots +
    \reg_P \kernl_P\!\brac{\ins, \ins, \param_P} \approx
    \feat_{\text{cat}} \Reg \feat_{\text{cat}}\T,
\end{equation}
where $\Reg = \diag{\sbrac{\reg_1 \mathbf{1}, \reg_2 \mathbf{1}, \ldots, \reg_P
        \mathbf{1}}}$ and each vector of $\mathbf{1}$ is the same dimension as
the corresponding basis. Here the regression regularizer, or weight prior
variance in Equations \eqref{eq:slm_prior} and \eqref{eq:glm_prior}, acts as a
kernel mixing/amplitude parameter.  Kernel products also have an equivalent
representation with basis functions (outer product of bases), however we have
not yet implemented this.

We also support \emph{partial application} of basis functions to certain
dimensions of the inputs, $\ins$, while also allowing concatentation, e.g,
\begin{equation}
    \feat_{\text{cat,partial}} = \sbrac{
        \func{\featsym}{\ins_{1:10}, \param_1}{1},
        \func{\featsym}{\ins_{5:D}, \param_2}{2},
        \ldots,
        \func{\featsym}{\ins, \param_P}{P}
    } 
\end{equation}
Where the subscript of $\ins$ denotes (arbitrary) column slices. See
\revrand{}'s documentation on how to use this features.


\section{Experiments}

In this section we apply \revrand{}'s standard linear model (SLM) and
generalized linear model (GLM) to a few standard regression and classification
datasets to ascertain how they perform relative to other standard machine
learning algorithms for these prediction tasks.


\subsection{Boston Housing Regression}

In Table \ref{tab:bostonhousing} we compare \revrand{}'s SLM and GLM on the
Boston Housing dataset. This is a regression problem where the aim is to
predict house prices in suburbs in Boston based on attributes of the buildings,
the region and the surrounding environment. There are 506 samples and 13
attributes in total.

\begin{table}[tb]

    \centering
    \caption{Regression performance on the Boston Housing dataset.}
    \label{tab:bostonhousing}
    \begin{tabular}{r|c c}
        \textbf{Algorithm} & \textbf{R-square} & \textbf{MSLL} \\
        \hline
        \emph{SLM} & 0.9018 (0.0134) & -1.504 (0.1191) \\
        % \emph{GLM} & 0.8340 (0.0491) & -0.8763 (0.1242) \\
        \emph{GLM} & 0.8411 (0.0491) & -0.9209 (0.1530) \\
        GP & \textbf{0.9027 (0.0137)} & \textbf{-1.1792 (0.1581)} \\
        RF & 0.8467 (0.0709) & N/A \\ 
        SVM & 0.6295 (0.0863) & N/A \\
        \hline
    \end{tabular}

\end{table}

For this experiment we use 5 fold cross validation, and use R-square and mean
standardised log-loss (MSLL) to evaluate the predictions on the held-out folds.
The SLM and GLM use random radial basis fourier features (RBF kernel
approximations) with 400 unique components (800 bases in total) for the SLM,
and 100 components (200 bases) for the GLM, both concatenated with a linear
basis. We found the GLM performed better with more stochastic gradient
iterations with fewer bases. We evaluate 100 random starts for the SLM and 500
for the GLM of the hyperparameters and then optimize from the best evaluation.
We use automatic relevance determination (ARD), that is, one length-scale per input
dimension for the RBF bases.

We compare against Scikit Learn \citep{scikit-learn} implementations of
Gaussian processes, support vector machines (SVM), and random forests (RF).
The GP is given the same kernel as the SLM and GLM, that is an additive ARD-RBF
and dot-product kernel. The GP's kernel amplitudes and it's RBF length-scales
are initialised at 1, the RF uses 40 decision trees, and the SVM uses a RBF
kernel, where the length-scale is chosen by a grid search with
cross-validation. The exact experiment can be replicated in the
\texttt{boston\_housing.ipynb} jupyter notebook.

As we would expect the GP yields the best performance, closely followed by the
SLM. The GLM and RF perform similarly, and the SVM the worst. 


\subsection{Handwritten Digits Classification}

We next test \revrand{}'s GLM on a classification dataset. We chose the USPS
handwritten digits binary classification task, also presented in
\citet{Rasmussen2006}. Here the aim is to classify the digits `3' and `5' from
$16\times16$ pixel black and white images. There are 767 training instances,
and 773 testing instances. We use log-loss and percent error as validation
scores.

\begin{table}[bt]

    \centering
    \caption{Binary classification performance on the USPS digits dataset for
        digits `3' and `5'.} 
    \label{tab:handwritten}.
    \begin{tabular}{r|c c}
        \textbf{Algorithm} & \textbf{Log-loss} & \textbf{Error (\%)} \\
        \hline
        \emph{GLM} & 0.1138 & \textbf{2.07} \\
        Logistic & 0.1734 & 3.62 \\
        SVM & \textbf{0.1003} & 6.99 \\
        GP &  0.1405 & 2.59 \\
        RF & 0.1368 & 2.72 \\
        \hline
    \end{tabular}

\end{table}

The GLM uses a Bernoulli likelihood with a logistic activation function, it
also uses RBF random fourier features with 800 components (1600 bases), and
again 500 random initialisations of the length scale (isotropic) and
regularizer with optimization proceeding from the best.

We compare to a SVM, RF with 40 estimators, logistic regressor with our same
random fourier features (un-optimized), and a GP classifier with the same
likelihood as our GLM, and using a Laplace posterior approximation. Again, all
of these are implementations in Scikit Learn \citep{scikit-learn}. The GP and
SVM also use RBF kernels, and the SVM uses a grid search and cross validation
to select the length scale. The results are summarized in Table
\ref{tab:handwritten}.

Interestingly, the SVM obtains the best log-loss, but has the worst error. The
GLM obtains the best error rate, and the second best log-loss, with the GP and
RF also performing well.


\subsection{SARCOS Regression}

The last experiment is a large-scale regression task presented in
\citet[Chapter 8]{Rasmussen2006}. The task is to predict the joint torque in a
SARCOS robot arm from all of the 7 joint states (position, angular velocity and
acceleration), 21 dimensions in all. There are 44,484 training examples, and
4449 test points. This dataset is too large for a typical GP, and so is a good
test for \revrand{}'s GLM using stochastic gradients\footnote{We have not yet
    run the SLM on this full dataset, the SLM's codebase needs to be modified
    to create the automatic relevance determination length-scales more
    efficiently, see issue \#130.}.

\begin{table}[tb]

    \centering
    \caption{Regression performance on the larger SARCOS robotic arm dataset.
        This table (apart from the GLM) has been reproduced from
        \citet{Rasmussen2006}.}
    \label{tab:sarcos}
    \begin{tabular}{r|c|c c}
        \textbf{Algorithm} & \textbf{Approximation size} & \textbf{SMSE} &
        \textbf{MSLL} \\
        % \hline
        % \emph{SLM (10000)} & 256 (ORF) & 0.0249 & -1.8591 \\
        % & 256 (RRF) & 0.0262 & -1.8480 \\
        % & 512 (ORF) & 0.0210 & -1.9590 \\
        % & 512 (RRF) & 0.0201 & -1.9730 \\
        \hline
        \emph{GLM} & 256 & 0.0358 & -1.6644 \\
        & 512 & 0.0252 & -1.8397 \\
        & 1024 & 0.0207 & -1.9341 \\
        & 2048 & 0.0171 & -2.0243 \\
        & 8192 & 0.0152 & -1.9804 \\
        \hline
        \emph{SD} & 256 & 0.0813 (.0198) & -1.4291 (.0558) \\
        & 512 & 0.0532 (.0046) & -1.5834 (.0319) \\
        & 1024 & 0.0398 (.0036) & -1.7149 (.0293) \\
        & 2048 & 0.0290 (.0013) & -1.8611 (.0204) \\
        & 4096 & 0.0200 (.0008) & -2.0241 (.0151)\\
        \hline
        \emph{SR} & 256 & 0.0351 (.0036) & -1.6088 (.0984) \\
        & 512 &  0.0259 (.0014) & -1.8185 (.0357) \\
        & 1024 & 0.0193 (.0008) & -1.9728 (.0207) \\
        & 2048 & 0.0150 (.0005) & -2.1126 (.0185) \\
        & 4096 & 0.0110 (.0004) & -2.2474 (.0204) \\
        \hline
        \emph{PP} & 256 & 0.0351 (.0036) & -1.6940 (.0528) \\
        & 512 & 0.0259 (.0014) & -1.8423 (.0286) \\
        & 1024 & 0.0193 (.0008) & -1.9823 (.0233) \\
        & 2048 & 0.0150 (.0005) & -2.1125 (.0202) \\
        & 4096 & 0.0110 (.0004) & -2.2399 (.0160) \\
        \hline
        \emph{BCM} & 256 & 0.0314 (.0046) & -1.7066 (.0550) \\
        & 512 & 0.0281 (.0055) & -1.7807 (.0820) \\
        & 1024 & 0.0180 (.0010) & -2.0081 (.0321) \\
        & 2048 & 0.0136 (.0007) & -2.1364 (.0266) \\
        \hline
    \end{tabular}

\end{table}

We summarise the results in Table \ref{tab:sarcos}, where we have reproduced
the results of the other algorithms from \citet{Rasmussen2006}. SD is a regular
GP using a random subset of data, SR is the subset of regressors method, PP
projected processes and BCM the Bayesian committee machine.

We ran the GLM with 1 million iterations of stochastic gradients (with a batch
size of 10), with 500 preliminary iterations with random values for the
algorithm parameters and hyperparameters. A RBF with automatic relevance
determination was used, and the inputs, $\ins$ were standardised to be
consistent with the kernels used in the original experiment.

In Table \ref{tab:sarcos} the `Approximation size' column refers the number of
unique components (half the number of bases) for the GLM, the number of data
points for SD, partitions for the BCM, and the number of inducing points for SR
and PP. We have only run the GLM once for each setting of the number of
components at this stage (because of the time required to run 1 million 
iterations on a single machine), we will present more comprehensive results in
the future.

The preliminary results in Table \ref{tab:sarcos} show that the GLM is
competitive with the other methods (not quite as good as SR, PP and BCM for
large approximation sizes), and is better than SD. It is worth noting though
that the GLM has a computational complexity of \bigo{ND} per iteration and the
SLM \bigo{ND^3}, whereas SD is \bigo{D^3} and SR, PP are \bigo{D^2 N} and the
BCM is the same cost as running $D$ independent GPs (each taking a portion of
N). Hence, we are able to use larger approximation sizes than the corresponding
methods.

\bibliography{report}

\end{document}
