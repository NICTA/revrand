\documentclass[11pt, oneside]{article}

% Packages
\usepackage{sectsty}
\usepackage{hyperref}
\usepackage{xcolor}
\usepackage{mathtools, amsfonts}

% Package config
\sectionfont{\normalfont\sffamily\bfseries}
\subsectionfont{\normalfont\sffamily\bfseries}
\hypersetup{colorlinks,
    linkcolor={red!50!black},
    citecolor={blue!50!black},
    urlcolor={blue!80!black}
}

\usepackage[backend=bibtex, style=numeric, firstinits=true]{biblatex}
\bibliography{report}

% Notation and macros
%% Define bracket commands (normal, square and curly).
\newcommand{\brac} [1]  {\ensuremath{\left({#1}\right)}}
\newcommand{\sbrac}[1]  {\ensuremath{\left[{#1}\right]}}
\newcommand{\cbrac}[1]  {\ensuremath{\left\{{#1}\right\}}}
\newcommand{\abrac}[1]  {\ensuremath{\left\langle{#1}\right\rangle}}


%% Symbols

% General
\newcommand{\test}       {\ensuremath{^{*}}}
\newcommand{\testT}      {\ensuremath{^{*\top}\!}}
\newcommand{\ttest}      {\ensuremath{^{**}}}
\newcommand{\real}   [1] {\ensuremath{\mathbb{R}^{#1}}}
\newcommand{\natu}   [1] {\ensuremath{\mathbb{N}^{#1}}}
\newcommand{\ident}  [1] {\ensuremath{\mathbf{I}_{#1}}}

% Variables
\newcommand{\targs}     {\ensuremath{y}}
\newcommand{\targ}      {\ensuremath{\mathbf{y}}}
\newcommand{\targd}     {\ensuremath{\mathbf{y}\test}}
\newcommand{\targdT}    {\ensuremath{\mathbf{y}\testT}}
\newcommand{\ins}       {\ensuremath{\mathbf{X}}}
\newcommand{\inss}      {\ensuremath{\mathbf{x}}}
\newcommand{\locs}      {\ensuremath{\mathbf{z}}}
\newcommand{\minibatch} {\ensuremath{\mathbf{B}}}
\newcommand{\feat}      {\ensuremath{\boldsymbol\Phi}}
\newcommand{\feats}     {\ensuremath{\boldsymbol\phi}}
\newcommand{\featfunc}  {\ensuremath{\phi}}
\newcommand{\weights}   {\ensuremath{\mathbf{w}}}
\newcommand{\var}       {\ensuremath{\sigma^2}}
\newcommand{\pomean}    {\ensuremath{\mathbf{m}}}
\newcommand{\pocov}     {\ensuremath{\mathbf{C}}}
\newcommand{\dpocov}    {\ensuremath{\boldsymbol{\Psi}}}
\newcommand{\dpocovs}   {\ensuremath{\Psi}}
\newcommand{\jacob}[1]  {\ensuremath{\mathbf{J}_{#1}}}
\newcommand{\hess}[1]   {\ensuremath{\mathbf{H}_{#1}}}
\newcommand{\reg}       {\ensuremath{\lambda}}
\newcommand{\param}     {\ensuremath{\theta}}
\newcommand{\lparam}    {\ensuremath{\gamma}}
\newcommand{\hyper}     {\ensuremath{\alpha}}
\newcommand{\lrate}     {\ensuremath{\eta}}

% Gaussian Process
\newcommand{\kernl}     {\ensuremath{k}}
\newcommand{\Kernl}     {\ensuremath{\mathbf{k}}}
\newcommand{\KERNL}     {\ensuremath{\mathbf{K}}}


%% Operations
\newcommand{\T}          {\ensuremath{^{\!\top}}}
\newcommand{\inv}        {\ensuremath{^{\text{-}1}}}
\newcommand{\deter}[1]   {\ensuremath{\left|{#1}\right|}}
\newcommand{\trace}[1]   {\ensuremath{\text{tr}\!\brac{#1}}}
\newcommand{\diag}[1]    {\ensuremath{\text{diag}\!\brac{#1}}}
\newcommand{\expec}[2]   {\ensuremath{\abrac{#2}_{\!{#1}}}}
\newcommand{\expece}[2]  {\ensuremath{\mathbb{E}_{#1}\!\sbrac{#2}}}
\newcommand{\evar} [2]   {\ensuremath{\mathbb{V}_{#1}\!\sbrac{#2}}}
\newcommand{\KL}[2]      {\ensuremath{\text{KL}\!\sbrac{{#1}\!\parallel\!{#2}}}}
\newcommand{\entropy}[1] {\ensuremath{\mathbb{H}\sbrac{#1}}}
\newcommand{\lnorm}[2]   {\ensuremath{\left\|{#2}\right\|_{{#1}}}}


%% Functions, PDFs etc
\newcommand{\func} [3] {\ensuremath{{#1}_{#3}\!\brac{{#2}}}}
\newcommand{\ffunc} [2] {\func{f}{#1}{#2}}
\newcommand{\activ} [1] {\func{g}{#1}{}}
\newcommand{\Prob}  [1] {\ensuremath{P\!\brac{#1}}}
\newcommand{\ProbC} [2] {\ensuremath{P\!\left({#1}\middle\vert{#2}\right)}}
\newcommand{\iProb}  [1] {\ensuremath{P\inv\!\brac{#1}}}
\newcommand{\iProbC} [2] {\ensuremath{P\inv\!\left({#1}\middle\vert{#2}\right)}}
\newcommand{\quant}  [1] {\ensuremath{Q\!\brac{#1}}}
\newcommand{\prob}  [1] {\ensuremath{p\!\brac{#1}}}
\newcommand{\probC} [2] {\ensuremath{p\!\left({#1}\middle\vert{#2}\right)}}
\newcommand{\qrob}  [1] {\ensuremath{q\!\brac{#1}}}
\newcommand{\qrobC} [2] {\ensuremath{q\!\left({#1}\middle\vert{#2}\right)}}
\newcommand{\gaus}  [1] {\ensuremath{\mathcal{N}\!\brac{#1}}}
\newcommand{\gausC} [2] {\ensuremath{\mathcal{N}\!\left({#1}\middle\vert{#2}\right)}}
\newcommand{\bern}  [1] {\ensuremath{\textrm{Bern}\!\brac{#1}}}
\newcommand{\bernC} [2] {\ensuremath{\textrm{Bern}\!\left({#1}\middle\vert{#2}\right)}}
\newcommand{\kfunc} [2] {\ensuremath{\kernl\!\brac{{#1}, {#2}}}}
\newcommand{\expon} [2] {\ensuremath{{#1}\!\times\!10^{#2}}}
\newcommand{\bigo}  [1] {\ensuremath{\mathcal{O}\!\brac{{#1}}}}
\newcommand{\elbo}      {\ensuremath{\mathcal{L}}}


%% Operators
\DeclareMathOperator*{\argmax}{\operatorname*{argmax}}
\DeclareMathOperator*{\argmin}{\operatorname*{argmin}}


% Preamble
\title{\sffamily{\emph{revrand}: Technical Report}}
\author{Daniel Steinberg, Louis Tiao \\
    Data61 $|$ CSIRO \\
    email: \texttt{\{firstname.lastname\}@data61.csiro.au}
}
\date{}

\begin{document}

\maketitle
\vspace{-0.5cm}
\noindent\makebox[\linewidth]{\rule{\linewidth}{0.8pt}}
\vspace{0.3cm}

\begin{abstract}
    This is a technical report on the \emph{revrand} software library. This
    library implements various Bayesian linear models (Bayesian linear
    regression), approximate Gaussian processes and generalised linear models.
    These algorithms have been implemented such that they can be used for
    large-scale inference by using stochastic gradients. All of the algorithms
    in \emph{revrand} use a unified feature composition framework, that allows
    for easy composition of positive-definite covariance functions and easy
    manipulation and combination of regression basis functions.
\end{abstract}

\tableofcontents

\section{Core Algorithms}

\subsection{Stochastic Gradients and Variational Objective Functions}

Stochastic Gradients is now a ubiquitous method for optimisation when a whole
dataset does not fit in memory, or when optimisation has to be distributed
amongst many computational nodes.

When an objective function factorises over data,
\begin{equation}
    \func{\ins, \param}{} = \sum_{\inss \in \ins} \func{\inss, \param}{},
\end{equation}
a regular gradient descent would perform the following iterations to minimise
the function w.r.t.\ $\param$,
\begin{equation}
    \param_k := \param_{k-1} - \lrate_k \sum_{\inss \in \ins}
    \nabla_{\param} \func{\inss_n, \param}{}\!|_{\param = \param_{k-1}},
\end{equation}
where $\lrate_k$ is the learning rate (step size) at iteration $k$. Stochastic
gradients proposes the following update,
\begin{equation}
    \param_k := \param_{k-1} - \lrate_k \sum_{\inss \in \minibatch}
    \nabla_{\param} \func{\inss, \param}{}\!|_{\param = \param_{k-1}},
    \label{eq:sg}
\end{equation}
where $\minibatch \subset \ins$ is a mini-batch of the original dataset, where
$|\minibatch| \ll |\ins|$.

Unfortunately some objective functions to not entirely decompose over the data,
i.e. 
\begin{equation}
    \func{\ins, \param}{} = \sum_{\inss \in \ins} \func{\inss, \param}{}
    + \ofunc{g}{\param}{}.
    \label{eq:objwconst}
\end{equation}
Let $M = |\minibatch|$ and $N = |\ins|$, then we divide the contribution of the
constant term amongst the mini-batches in stochastic gradients,
\begin{equation}
    \param_k := \param_{k-1} - \lrate_k \sum_{\inss \in \minibatch}
    \nabla_{\param} \func{\inss, \param}{}\!|_{\param = \param_{k-1}}
    - \lrate_k \frac{M}{N} \nabla_{\param} 
    \ofunc{g}{\param}{}\!|_{\param = \param_{k-1}}.
    \label{eq:wsg}
\end{equation} 
This is particularly relevant for variational inference where the evidence
lower bound objective has a component independent of the data. For example, 
lets consider the model,
\begin{align}
    \text{Likelihood:} \quad &\prod^N_{n=1} \probC{\targs_n}{\param}, \\
    \text{prior:} \quad &\probC{\param}{\hyper},
\end{align}
where we want to learn the values of the hyper-parameters, $\hyper$. Minimising
negative log-marginal likelihood is a good objective in this instance, since we
don't care about the value(s) of $\param$,
\begin{equation}
    \argmin_\hyper - \log \int \prod^N_{n=1} \probC{\targs_n}{\param}
    \probC{\param}{\hyper} d \param.
    \label{eq:lml}
\end{equation}
There are two problems with this objective however, (1) it may not factor over
data and (2) the integral may be intractable, like if the prior and likelihood
are not conjugate. In variational inference we use Jensen's inequality to
lower-bound log-marginal likelihood with a tractable objective function called
the evidence lower bound (ELBO),
\begin{align}
    \log \probC{\targ}{\hyper} =& \log \int 
        \prod^N_{n=1} \probC{\targs_n}{\param} 
        \probC{\param}{\hyper} d \param \nonumber \\
        =& \log \int 
        \frac{\prod_n \probC{\targs_n}{\param} \probC{\param}{\hyper}}
        {\qrob{\param}} \qrob{\param} d \param \nonumber \\
        \geq& \int \qrob{\param} \log \sbrac{%
            \frac{\prod_n \probC{\targs_n}{\param} 
            \probC{\param}{\hyper}}{\qrob{\param}}}
        d \param
\end{align}
where $\qrob{\param}$ is an approximation of $\probC{\param}{\hyper}$ that 
makes inference easier. This can be re-written as,
\begin{equation}
    \elbo = \sum^N_{n=1} \expec{q}{\log\probC{\targs_n}{\param}} -
    \KL{\qrob{\param}}{\probC{\param}{\hyper}},
    \label{eq:elbo}
\end{equation}
which takes the form of Equation~\eqref{eq:objwconst}, and so we can weight the
Kullback-Leibler term like the constant term, $\ofunc{g}{\cdot}{}$, from
Equation~\eqref{eq:wsg} if we use stochastic gradients optimisation.
Furthermore, if $\qrob{\param} = \probC{\param}{\hyper}$ then the lower bound
is tight, and this will be equivalent to optimising log-marginal likelihood.


\subsection{Bayesian Linear Regression}

The first machine learning algorithm in revrand is a simple Bayesian linear
regressor of the following form,
\begin{align}
    \text{Likelihood:} \quad &\prod^N_{n=1} 
    \gausC{\targs_n}{\feats_n\T\weights, \var}, \\
    \text{prior:} \quad &\gausC{\weights}{\mathbf{0}, \reg\ident{D}},
\end{align}
where $\feats_n := \ofunc{\featfunc}{\inss_n, \param}{}$ is a feature function
that $\featfunc: \real{d} \to \real{D}$. This is, for all intents and purposes
the same algorithm described in~\cite[Chapter 2]{Rasmussen2006}.

We then:
\begin{itemize}
    \item Solve analytically for the posterior over weights, $\weights | \targ
        \sim \gaus{\pomean, \pocov}$,
    \item Optimise $\var, \reg$ and $\param$ w.r.t.\ log-marginal likelihood,
        which is tractable,
    \item Use the (analytically solvable) predictive distribution
        \begin{equation}
            \probC{\targs\test}{\targ, \ins, \inss\test} = \int
            \gausC{\targs\test}{\feats\testT\weights, \var}
            \gausC{\weights}{\pomean, \pocov} d\weights
        \end{equation}
        for query inputs, $\inss\test$.
\end{itemize}

It is actually easier to use the ELBO form with stochastic gradients for
learning the parameters of this algorithm, rather than log-marginal likelihood.
This is simply because it is plainly in the same form as Equation
\eqref{eq:objwconst}, though it would give the same result as log-marginal
likelihood, since no approximation is made about the posterior.

We have not implemented a stochastic gradient version of this algorithm since
it still requires a matrix solve involving a $D \times D$ matrix, and so is
\bigo{D^3} in complexity, per iteration. This is true even if we optimise the
posterior covariance directly (or some triangular form). The GLM presented in
the next section circumvents this issue, and is more suited to really large $N$
and $D$ problems.


\subsection{Bayesian Generalised Linear Models}

A modification of the GLM in~\cite{gershman2012} to make it work with
stochastic gradients.

\subsection{Large Scale Gaussian Process Approximation}

\subsubsection{With Bayesian Linear Regression}

i.e.\ A la Carte~\cite{yang2014}

we still need to invert a $D \times D$ matrix with stochastic gradients, so
\bigo{D^3}, but that is where the GLM comes in.

\subsubsection{With Bayesian GLMs}

We only have to invert $K$ diagonal matrices of dimension $D \times D$, so
\bigo{KD}.

\section{Feature Composition Framework}

\section{Future Functionality}

\printbibliography%

\end{document}
